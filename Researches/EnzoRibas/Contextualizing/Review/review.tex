\documentclass{article}
\usepackage[utf8]{inputenc}
\usepackage[brazil]{babel}
\usepackage{hyperref}

\begin{document}

\title{Revisão de Artigos para Contextualização do Projeto ATOM\\(Robótica e Visão Computacional)}
\author{Enzo Ribas}
\date{2025}

\maketitle
\section*{Artigo 1: Computer Vision Enabled Smart Surveillance for Urban Traffic Control}

\subsection*{Objetivo do Artigo}
O trabalho propõe e avalia um sistema inteligente de monitoramento de tráfego urbano baseado em visão computacional e aprendizado profundo, integrado com IoT e computação em borda. O objetivo é detectar, classificar e rastrear veículos e pedestres em tempo real, otimizando o fluxo viário e melhorando a segurança.

\subsection*{Metodologia}

O sistema é estruturado em múltiplas camadas:
\begin{itemize}
    \item \textbf{Aquisição de Dados:} Câmeras de alta resolução, sensores IoT e GPS coletam informações sobre tráfego, velocidade e ocupação de faixas.
    \item \textbf{Processamento e Armazenamento:} Dados pré-processados na nuvem e na borda para limpeza, extração de atributos e atualização de modelos.
    \item \textbf{Predição e Decisão:} Modelos de Machine Learning (Random Forest, SVM, CNNs, LSTMs) preveem congestionamento e ajustam sinais dinamicamente.
    \item \textbf{Controle Inteligente de Semáforos:} Uso de Reinforcement Learning e lógica fuzzy para ajustar o tempo de sinais com prioridade para veículos de emergência.
    \item \textbf{Interface e Monitoramento:} Painéis para autoridades e aplicativos móveis para usuários.
\end{itemize}

\textbf{Tecnologias-chave:}
\begin{itemize}
    \item YOLO, SSD e Faster R-CNN para detecção de objetos.
    \item Edge computing para reduzir latência em 40\%.
    \item LPR (License Plate Recognition) para fiscalização automática.
    \item V2I (Vehicle-to-Infrastructure) para priorização de transporte público e emergências.
\end{itemize}

\subsection*{Resultados Principais}

\begin{itemize}
    \item \textbf{Acurácia de detecção:} Veículos (95,2\%), pedestres (93,8\%), placas (88,9\%).
    \item \textbf{Previsão de congestionamento (LSTM):} 89,7\% de acurácia.
\end{itemize}

\textbf{Impactos positivos:}
\begin{itemize}
    \item Redução do índice de congestionamento em até 50\%.
    \item Redução do tempo médio de espera de 120s para 65s.
    \item Aumento de 25--40\% no fluxo de veículos.
    \item Resposta a emergências 50\% mais rápida.
    \item Processamento de vídeo em 25 FPS, permitindo operação em tempo real.
\end{itemize}

\subsection*{Contribuição para o Conhecimento}

O artigo demonstra como a fusão de visão computacional de última geração, IA adaptativa, IoT e computação distribuída pode criar sistemas de controle urbano altamente eficientes.

\textbf{Aplicações para o projeto ATOM:}
\begin{itemize}
    \item Uso de edge computing e modelos leves (YOLO, SSD) é crucial para robótica embarcada.
    \item Integração de múltiplos sensores e comunicação V2I pode inspirar a integração do braço com redes de sensores para feedback em tempo real.
    \item O enfoque em robustez sob condições adversas (variação de luz, clima, oclusão) é diretamente aplicável à operação de robôs em ambientes não controlados.
\end{itemize}

\subsection*{Limitações e Perspectivas Futuras}

\begin{itemize}
    \item Dependência da qualidade das câmeras e iluminação.
    \item Desafios com oclusão em tráfego denso.
    \item Necessidade de integração com visão térmica/3D para ambientes complexos.
\end{itemize}

\textbf{Próximos passos sugeridos:}
\begin{itemize}
    \item Uso de detecção 3D para maior precisão.
    \item Integração com infrared/thermal imaging para operação noturna.
    \item Aprendizado federado para melhorar os modelos sem comprometer a privacidade.
    \item Expansão para monitorar transporte multimodal.
\end{itemize}

\subsection*{Conclusão}

O estudo é um exemplo claro de aplicação de tecnologias emergentes em um problema real, apresentando resultados expressivos. Para o ATOM, serve como prova de que a combinação de visão computacional, IA e processamento distribuído pode gerar sistemas robustos e escaláveis --- algo que pode ser replicado em braços robóticos para ambientes industriais ou educacionais.

\bigskip

\section*{Artigo 2: Computer Vision Applications in Defense and Medical Imaging}

\subsection*{Objetivo do Artigo}

O estudo investiga como a Visão Computacional, integrada a Inteligência Artificial, Deep Learning e Computação em Borda, está transformando setores críticos:

\begin{itemize}
    \item \textbf{Saúde:} diagnóstico por imagem, cirurgia assistida por robôs e descoberta de medicamentos.
    \item \textbf{Defesa:} vigilância autônoma, reconhecimento de ameaças, drones inteligentes e manutenção preditiva.
\end{itemize}

A proposta é explorar avanços, desafios e tendências futuras, destacando a interseção entre automação inteligente e tomada de decisão em tempo real.

\subsection*{Metodologia e Abordagem}

O artigo é estruturado em seis seções:
\begin{enumerate}
    \item \textbf{Revisão de Literatura:} Consolida pesquisas recentes sobre aplicação de CNNs, IA generativa, agentes RAG (Retrieval-Augmented Generation) e integração com Big Data.
    \item \textbf{Aplicações em Saúde:} Desde análise automatizada de raios-X, tomografias e ressonâncias até cirurgia robótica com feedback visual em tempo real.
    \item \textbf{Aplicações em Defesa:} Uso em UAVs, reconhecimento facial, veículos terrestres autônomos e integração multimodal de sensores.
    \item \textbf{Desafios:} Viés de dados, privacidade, ``caixa-preta'' dos modelos e limitações de desempenho em borda.
    \item \textbf{Tendências Futuras:} IA explicável, integração de dados multimodais e colaborações civis-militares para acelerar a adoção.
    \item \textbf{Conclusão:} Reforça que a visão computacional está evoluindo para um papel central na segurança e saúde globais.
\end{enumerate}

\subsection*{Resultados e Exemplos Destacados}

\textbf{Medicina:}
\begin{itemize}
    \item CNNs para segmentação de tumores e detecção precoce de doenças como retinopatia diabética e câncer de pele.
    \item Edge AI para diagnóstico portátil em áreas remotas, sem depender da nuvem.
    \item Visão computacional em robótica cirúrgica, aumentando precisão e reduzindo erros.
\end{itemize}

\textbf{Defesa:}
\begin{itemize}
    \item Drones autônomos para reconhecimento e mapeamento de áreas hostis.
    \item Sistemas de visão para mísseis guiados e veículos autônomos militares.
    \item Reconhecimento facial com grandes bases biométricas, aumentando acurácia e reduzindo falsos positivos.
    \item Manutenção preditiva de equipamentos via inspeção visual.
\end{itemize}

\textbf{Tecnologias-chave:}
\begin{itemize}
    \item IA generativa e agentes RAG para simulação e planejamento de cenários militares.
    \item Integração multimodal (imagem, áudio, sensores ambientais) para aumentar consciência situacional.
\end{itemize}

\subsection*{Contribuição para o Conhecimento}

O trabalho reforça que visão computacional não é mais um recurso isolado, mas parte de ecossistemas inteligentes com impacto direto na tomada de decisão crítica.

\textbf{Paralelos para o projeto ATOM:}
\begin{itemize}
    \item Abordagem multimodal pode ser aplicada a braços robóticos que integrem câmeras, sensores de força e feedback háptico.
    \item Edge computing garante resposta em tempo real mesmo com hardware embarcado.
    \item Técnicas de IA explicável e tratamento de dados sensíveis são fundamentais para aplicações em ambientes acadêmicos e industriais.
\end{itemize}

\subsection*{Limitações e Desafios}

\begin{itemize}
    \item Dependência de dados de qualidade para evitar viés.
    \item Necessidade de explicabilidade em decisões críticas.
    \item Restrições de processamento em dispositivos de borda com recursos limitados.
    \item Questões éticas e legais, especialmente em defesa e dados médicos.
\end{itemize}

\subsection*{Perspectivas Futuras}

\begin{itemize}
    \item IA Explicável para aumentar confiança em decisões médicas e militares.
    \item Sistemas Multimodais combinando visão, áudio e sensores diversos.
    \item Colaboração Civil-Militar para acelerar a inovação e aumentar a eficiência.
    \item Prompt Engineering + IA Generativa para adaptar sistemas em tempo real.
\end{itemize}

\subsection*{Conclusão}

O artigo posiciona a visão computacional como um pilar tecnológico estratégico para o futuro da saúde e da defesa. A integração de deep learning, edge computing e IA generativa cria sistemas mais rápidos, precisos e adaptáveis, mas exige atenção a ética, privacidade e robustez. Para projetos como o ATOM, esta referência mostra o potencial de criar soluções robóticas inteligentes que operem com alta precisão e segurança em cenários complexos.

\end{document}